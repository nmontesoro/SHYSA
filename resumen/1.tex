\chapter{Principios de la higiene del trabajo y seguridad en el trabajo}

\section{Condiciones de trabajo. Ambiente de trabajo. Relación
  hombre-tarea-ambiente}

Las \acrfull{cymat} están compuestas por factores socio-técnicos y
organizacionales del proceso de producción y \glspl{factor-de-riesgo} del medio
ambiente laboral. La combinación de estos factores determina la carga global
de trabajo que afecta la vida y la salud física, psicológica y mental de los
trabajadores. La salud ocupacional busca promover el bienestar físico, mental
y social de los trabajadores, evitar daños causados por las condiciones de
trabajo, protegerlos de riesgos y adaptar el trabajo al hombre y cada hombre a
su trabajo. Los factores que afectan el bienestar de los trabajadores se pueden
agrupar en distintas categorías.

\subsection{Factores del medio ambiente de trabajo}

Hay cuatro categorías de factores que afectan el bienestar de los trabajadores
y que surgen de los procesos y actividades del puesto de trabajo:

\begin{itemize}
	\item Factores físicos: relacionados con el ambiente laboral, como ruido,
	      vibraciones, temperaturas extremas, radiaciones, presiones anormales e
	      iluminación.
	\item Factores químicos: originados por el manejo o exposición a elementos
	      químicos venenosos, irritantes o corrosivos, que pueden afectar
	      directamente el organismo.
	\item Factores biológicos: asociados a actividades que implican la
	      presencia de agentes patógenos, como virus, hongos, bacterias o
	      parásitos.
	\item Factores ergonómicos: derivados de una incorrecta ergonomía laboral
	      que puede llevar al desarrollo de trastornos musculoesqueléticos, como
	      sobrecarga postural, repetitividad de movimientos y manipulación de
	      cargas y tiempos (ver \textit{\gls{ergonomia}}).
\end{itemize}

\subsection{Factores de infraestructura}

Los factores de infraestructura están relacionados con la \gls{infraestructura}
de los edificios, equipos y herramientas utilizados en el lugar de trabajo.
Estos factores pueden generar peligros y riesgos para los trabajadores. Los
principales factores incluyen máquinas y herramientas, edificios, mobiliario,
instalaciones de servicio y equipamiento social.

\subsection{Organización y contenido del trabajo}

Refiere a varios aspectos del régimen laboral, que incluyen los turnos (fijos o
rotativos), el horario de trabajo (diurno o nocturno, horas extras, fines de
semana), el régimen salarial (fijo, por objetivos o por producción), la
programación de pausas y descansos, y la carga mental, que incluye la
naturaleza de las tareas (dinámicas, repetitivas o monótonas), la presión del
tiempo y el acoso laboral.

\subsection{Aspectos asistenciales y sociales}

Los servicios asistenciales se refieren a la atención en temas de higiene y
seguridad, medicina del trabajo y promoción social para los trabajadores. Los
beneficios sociales incluyen transporte, recreación y cultura.

\section{Higiene del trabajo. Definición y principios.}

\subsection{Higiene del trabajo}

La Higiene del Trabajo, también conocida como higiene industrial, es la
ciencia que se encarga de anticipar, identificar, evaluar y controlar los
riesgos que pueden poner en peligro la salud y bienestar de los trabajadores
en su lugar de trabajo y su posible repercusión en el medio ambiente. Sus
objetivos son eliminar las causas de las enfermedades, reducir los efectos
perjudiciales del trabajo, prevenir el agravamiento de enfermedades y lesiones,
mantener la salud de los trabajadores y aumentar la productividad con un buen
ambiente de trabajo. La práctica de la higiene industrial consta de tres etapas:
identificación de posibles peligros para la salud, evaluación de los riesgos y
gestión de riesgos mediante el desarrollo e implantación de controles
específicos para eliminar o reducir los riesgos.

\subsection{Enfermedad profesional}

Las enfermedades profesionales son causadas por agresores químicos, físicos o
biológicos presentes en el ambiente laboral y pueden aparecer lentamente,
causando incapacidad o incluso la muerte. Estas enfermedades son específicas de
cada profesión y actividad, y si no se evitan, controlan o tratan a tiempo,
pueden tener graves consecuencias. La legislación argentina cuenta con un
\textbf{Listado de Enfermedades Profesionales} que identifica los agentes de
riesgo, los cuadros clínicos, las exposiciones y las actividades en las que se
producen estas enfermedades.

\section{Seguridad en el trabajo. Definiciones y principios.}

\subsection{Seguridad en el trabajo}

La seguridad en el trabajo es un conjunto de principios y normas que buscan
prevenir accidentes y controlar riesgos en el ambiente laboral. Su objetivo es
mejorar las condiciones de trabajo para evitar lesiones, muertes y daños a
equipos, materiales y medio ambiente. También busca reducir costos operativos y
mejorar la imagen de la empresa, aumentando el rendimiento laboral. Se busca
contar con un sistema estadístico para detectar el avance o disminución de los
accidentes y su causa, y contar con los medios necesarios para implementar un
plan de seguridad.

\subsection{Accidente de trabajo}

Un accidente de trabajo es un evento súbito y violento que ocurre durante o en
relación con el trabajo, o en el trayecto entre el hogar y el lugar de trabajo,
que puede causar daños a la salud o incluso la muerte del trabajador. La
definición se encuentra en la \acrfull{lrt}, y excluye los accidentes causados
por dolo del trabajador o fuerza mayor ajena al trabajo.

\subsection{Incidente}

Es un acontecimiento no deseado, que bajo circunstancias ligeramente
diferentes, podría haber resultado en lesiones a las personas, daño a la
propiedad o pérdida para el proceso. Es un evento que puede dar como resultado
un accidente o tiene el potencial para ocasionarlo.

\subsection{Severidad del daño. Naturaleza de las lesiones}

La severidad del daño se refiere al grado de lesión o incapacidad que sufre un
trabajador como consecuencia de un accidente laboral o enfermedad profesional.
Según la legislación nacional, se clasifica en diferentes tipos de incapacidad
laboral temporal o permanente, que van desde una lesión temporal de hasta 12
meses, hasta la gran invalidez, en la que el trabajador necesita asistencia
para realizar sus actos esenciales. El objetivo de clasificar la severidad del
daño es poder otorgar la compensación adecuada a los trabajadores afectados.

\subsection{Responsabilidades en higiene y seguridad en el trabajo}

\begin{itemize}
	\item Dirección de la empresa
	      \begin{itemize}
		      \item Cumplir con legislación
		      \item Definir política de \acrfull{hyst}
		      \item Delimitar responsabilidades
		      \item Establecer normas de seguridad
		      \item Proporcionar lugares seguros de trabajo
		      \item Proveer equipos y \acrfull{ppe}
		      \item Implementar un plan de capacitación en prevención de riesgos
		      \item Realizar o gestionar los exámenes de salud de los
		            trabajadores
	      \end{itemize}
	\item Trabajadores
	      \begin{itemize}
		      \item Cumplir con normas de \acrshort{hyst}
		      \item Someterse a los exámenes de salud
		      \item Asistir a los cursos de capacitación
		      \item Colaborar en la detección de factores de riesgo
		      \item Informar los accidentes e incidentes
	      \end{itemize}
	\item Supervisión
	      \begin{itemize}
		      \item Incultar hábitos seguros de trabajo
		      \item Controlar cumplimiento de normas de seguridad
		      \item Detectar factores de riesgo de accidentes y enfermedades, e
		            investigarlos
	      \end{itemize}
	\item Medicina laboral
	      \begin{itemize}
		      \item Efectuar los exámenes de salud
		      \item Colaborar con el \acrfull{sshyst} en la investigación de
		            accidentes y factores de riesgo
		      \item Efectuar el seguimiento de los accidentados y afectados
		      \item Confeccionar y actualizar legajos médicos
	      \end{itemize}
	\item Higiene y seguridad
	      \begin{itemize}
		      \item Establecer objetivos y elaborar programas de \acrshort{hyst}
		      \item Controlar cumplimiento de normas, adoptando medidas
		            preventivas
		      \item Especificar características y conservación de los
		            \acrshort{ppe}, de almacenamiento y transporte de material,
		            producción, distribución, etc.
		      \item Redactar textos para etiquetado de sustancias nocivas
		      \item Elaborar reglamentaciones, normas y procedimientos
		      \item Llevar estadísticas
		      \item Registrar todas las evaluaciones de los contaminantes
		            ambientales existentes
	      \end{itemize}
\end{itemize}

\section{Evaluación y control de riesgos}

\subsection{Evaluación de riesgos. Conceptos principales.}

La \textbf{evaluación de riesgos} es un proceso para identificar y estimar la
magnitud de los riesgos asociados a una tarea o proceso, con el objetivo de
planificar y adoptar medidas preventivas para evitar accidentes y enfermedades
laborales. Para llevar a cabo la evaluación, es necesario identificar los
\textbf{\glspl{peligro}}, que son fuentes con potencial para causar lesiones y
deterioro de la salud, y que pueden estar relacionados con factores tangibles o
intangibles del ambiente laboral. El \textbf{\gls{riesgo}} se define como la
combinación de la \gls{probabilidad} de que ocurra un evento o exposición peligrosa y
la severidad de la lesión o deterioro de la salud que puede causar, y se
utiliza para establecer la necesidad de intervenir en una tarea o proceso y
mejorar el nivel de seguridad de los trabajadores.

\subsubsection{Método de las cinco ``M''}

El método de las 5M es una herramienta desarrollada por Toyota para encontrar
fallos en un sistema. Puede ser utilizada como regla para el abordaje
sistemático de los procesos y como herramienta complementaria para el análisis
de peligros y riesgos. Se trata de un \textbf{sistema de análisis estructurado}
que se basa en cinco pilares fundamentales: \textbf{Máquina, Método, Mano de
	obra, Medio ambiente y Materia prima}. Estos pilares permiten reconocer los
peligros a los que se pueden exponer los trabajadores y las posibles causas de
un problema. Al analizar cada pilar se pueden identificar los factores de
riesgo y tomar medidas preventivas para evitar accidentes y mejorar la
eficiencia del proceso.

\subsection{Métodos de evaluación de riesgo}

La mayoría de los métodos de evaluación de riesgos laborales utilizados no
pretenden determinar un valor real del riesgo, sino que se conforman con una
aproximación en términos de nivel, usando escalas arbitrarias. Estos métodos
se orientan a estimar el nivel de riesgo para situaciones en las cuales pueda
producirse un accidente. Se aplican métodos de evaluación específicos para los
factores físicos, químicos y ergonómicos, y se contemplan variables con impacto
a lo largo del tiempo. Los métodos de mayor aplicabilidad en situaciones de
accidentes se resumen en una tabla que incluye la valoración simple (ABC),
método general, método Fine, Steel, y Strohm y Opheim.

\subsubsection{Método general}

Se utilizan tablas de doble entrada de 3x3 o 5x5 para establecer el nivel de
riesgo a partir de estimaciones cualitativas de la gravedad del accidente y la
\gls{probabilidad} de ocurrencia. Para determinar la \textbf{gravedad} se consideran
las partes del cuerpo afectadas, la naturaleza del daño y la protección
suministrada por los elementos de protección personal. Para establecer la
\textbf{\gls{probabilidad} de ocurrencia} se considera la frecuencia de exposición al
peligro, la seguridad de los equipos y procesos, los posibles fallos en los
componentes de las instalaciones y las máquinas, entre otros factores.
Los \textbf{plazos y el tipo de intervención} en función del nivel de riesgo
son establecidos por cada organización, pero un modelo ampliamente utilizado
considera acciones sencillas para eliminar o neutralizar el riesgo en un período
flexible para el nivel tolerable y acciones para eliminar o neutralizar el
riesgo en un período definido para el nivel moderado. El nivel sustancial
requiere la eliminación o neutralización del riesgo antes de comenzar el
trabajo, mientras que el nivel intolerable requiere la eliminación o
neutralización del riesgo y la prohibición de la ejecución del trabajo si no es
posible hacerlo.

\subsubsection{Método Fine}

Es un método para evaluar el riesgo en una tarea o actividad en función de la
gravedad $g$, la frecuencia $f$ y la \gls{probabilidad} $p$ de un evento no deseado.
El método desvincula la frecuencia de exposición de la variable probabilidad y
asigna criterios para la elección de las categorías de gravedad y frecuencia.
La valoración del riesgo se obtiene multiplicando la gravedad, la frecuencia y
la probabilidad ($p\cdot g \cdot f$), y se establecen cuatro niveles de riesgo,
desde muy alto hasta aceptable, con medidas de detención inmediata, corrección
urgente o necesaria, o la posibilidad de omitir la corrección.

\subsection{Control de los riesgos}

\subsubsection{Principios de la acción preventiva}

Los principios de la acción preventiva son:

\begin{itemize}
	\item Evitar los riesgos.
	\item Evaluar los riesgos que no se puedan evitar.
	\item Combatir los riesgos en su origen.
	\item Adaptar el trabajo a la persona, en lo que respecta a la concepción 
        de los puestos de trabajo y a la elección de los equipos y los métodos 
        de trabajo y de producción, con miras a atenuar el trabajo monótono y 
        repetitivo y a reducir los efectos del mismo en la salud.
	\item Sustituir lo peligroso por lo que entrañe poco o ningún peligro.
	\item Planificar la prevención, integrando la técnica, la organización del 
        trabajo, las condiciones de trabajo, las relaciones sociales y la 
        influencia de los factores ambientales.
	\item Adoptar medidas que antepongan la protección colectiva a la 
        individual.
	\item Dar las debidas instrucciones a los trabajadores.
	\item Tomar en consideración las capacidades profesionales de los 
        trabajadores en materia de seguridad y de salud en el momento de 
        encomendarles las tareas.
	\item Adoptar las medidas necesarias a fin de garantizar que sólo los 
        trabajadores que hayan recibido información suficiente y adecuada 
        puedan acceder a las zonas de riesgo grave y específico.
\end{itemize}

\subsubsection{Medidas de control de riesgos}

Es importante tener en cuenta la jerarquización de los controles implementados 
para garantizar la seguridad y salud de los trabajadores en el lugar de 
trabajo. El estándar ISO-IRAM 45001 propone una jerarquización de controles que
incluye la \textbf{eliminación} de peligros, la \textbf{sustitución} de lo 
peligroso por lo menos peligroso, los \textbf{controles de ingeniería} y la 
reorganización del trabajo, los \textbf{controles administrativos}, y el 
\textbf{\acrshort{ppe}}.

También se mencionan otras clasificaciones, como la \textbf{clasificación por 
    variable atendida}, que agrupa las medidas según sean de prevención o de 
protección, y la \textbf{clasificación por localización}, según intervengan 
sobre la fuente, la propagación en el medio o la persona. 

\section{Tratamiento de los accidentes laborales}

\subsection{Causas de los accidentes}

Las causas más comunes de los accidentes laborales son la falta de 
capacitación, el incumplimiento de las normas de seguridad, equipos y 
herramientas defectuosas, la fatiga, el estrés y la presión, la falta de 
supervisión, las condiciones de trabajo peligrosas y la falta de 
\acrshort{ppe}. Estas causas se pueden agrupar en tres categorías: 
\textbf{condición insegura} (fallas o desviaciones del equipo o lugar de 
trabajo), \textbf{acto inseguro} (aquel que por ser realizado u omitido hace 
posible el accidente) y \textbf{factor humano} (característica mental o física
que puede exponer al trabajador).

\subsubsection{Teoría de la causalidad múltiple}

La teoría de la causalidad múltiple explica que para cada accidente laboral hay
múltiples factores, causas y subcausas que contribuyen a su aparición. Estos 
factores se dividen en dos categorías: de comportamiento (relacionados con el 
trabajador) y ambientales (relacionados con el entorno laboral). Esta teoría 
destaca que rara vez un accidente es causado por una sola acción o factor.

\subsection{Investigación de los accidentes}

La investigación de accidentes busca descubrir las causas de los accidentes y 
neutralizar los riesgos desde su origen para prevenir su repetición. Tiene 
objetivos directos (conocer los hechos y sus causas) y preventivos 
(eliminar las causas y aprovechar la experiencia para prevenir).

La \textbf{Pirámide de Bird} es un modelo que describe la relación entre los 
accidentes laborales y los incidentes que los preceden. Fue desarrollado por el
investigador de seguridad industrial Frank Bird Jr. en la década de 1960. 
Muestra que por cada accidente grave en el lugar de trabajo, hay una serie de 
eventos menos graves que lo preceden. En la base de la pirámide se encuentran 
los incidentes, que son eventos en los que no se produjo ningún daño o lesión, 
pero que podrían haberlo hecho. Por encima de los incidentes, están los 
accidentes leves, seguidos por los accidentes graves y, en la cima de la 
pirámide, están los accidentes fatales.

El modelo \textbf{destaca la importancia de la prevención y el control de los 
    incidentes para evitar accidentes más graves}. En este sentido, la 
Pirámide de Bird sugiere que si se controlan y previenen los incidentes, se 
puede evitar que ocurran accidentes graves o fatales en el lugar de trabajo.

\subsection{Métodos de investigación}

\subsubsection{Métodos simplificados}

Las normas de la \acrfull{oit} evalúan los accidentes de trabajo según cuatro 
factores: \textbf{forma del accidente, agente material, naturaleza de la lesión
    y ubicación de la lesión}. Sin embargo, este enfoque es limitado y no 
proporciona un método óptimo para investigar accidentes. Una forma más completa
de analizar un accidente es identificar la forma del accidente, la naturaleza
de la lesión, la parte del cuerpo afectada, la fuente de la lesión, el agente
del accidente, la parte del agente, la condición insegura y el acto inseguro.
Este enfoque proporciona una visión más detallada de las causas del accidente y
permite implementar medidas preventivas más efectivas.

\subsubsection{Método del árbol de causas}

La \acrfull{srt} promueve el Método del árbol de causas para investigar 
accidentes y prevenir futuros casos. Este método se basa en \textbf{encontrar 
    relaciones entre los hechos} que contribuyeron al accidente y \textbf{no en
    buscar culpables}. Se recopilan todos los datos relevantes sobre el 
accidente y se construye un árbol partiendo del suceso último, remontando 
sistemáticamente de hecho en hecho para identificar problemas de fondo que 
originaron las condiciones en las que sucedió el accidente. Las medidas 
preventivas adoptadas ayudan a evitar otros accidentes.

\section{Legislación}

La legislación es el conjunto de leyes que regulan la vida en un país y 
establecen las conductas aceptables o rechazables. Las leyes son dictadas por 
los legisladores y deben ser respetadas y cumplidas por todos los ciudadanos. 
En todo sistema jurídico las normas se organizan jerárquicamente a partir de la
Constitución, que es la norma más alta, hasta llegar a las sentencias 
judiciales. 

En Argentina, la Constitución Nacional es la norma jurídica más general, seguida
de los tratados con potencias extranjeras y las leyes nacionales, y las 
Constituciones Provinciales deben conformarse a la Constitución Nacional.

\subsection{Ley 19587 de Higiene y Seguridad en el Trabajo}

La ley nacional 19587 establece las condiciones de \acrshort{hyst} en toda la 
República Argentina, aplicable a cualquier establecimiento y actividad 
económica. Su objetivo es proteger a los trabajadores y prevenir accidentes y 
enfermedades laborales. Está reglamentada por el decreto 351/79 y otras normas 
complementarias que establecen especificaciones técnicas y requerimientos 
específicos de cumplimiento obligatorio para diferentes actividades y 
organizaciones.

\subsection{Ley 24557 de Riesgos del Trabajo}

La ley 24557 de Riesgos del Trabajo (\acrshort{lrt}) establece la creación de 
las \acrfull{art} 
y de la \acrshort{srt}. Tiene como objetivos reducir los siniestros laborales a
través de la prevención de riesgos, reparar daños derivados de accidentes y 
enfermedades profesionales y promover la recolocación de los trabajadores 
damnificados. La ley obliga a los empleadores a contar con cobertura de 
\acrshort{art} y a cumplir con la ley 19587. Las contingencias cubiertas son 
accidentes de trabajo, accidentes \textit{in itinere} y enfermedades 
profesionales, con prestaciones en especie y dinerarias para los empleados 
damnificados.

\subsection{Apéndice}

En términos generales, un \textbf{\gls{sistema}} se refiere a un conjunto de
elementos o componentes interconectados que trabajan juntos para lograr un 
objetivo común, caracterizado por la interdependencia entre sus componentes y 
con límites que lo separan del entorno externo. En todos los casos, el objetivo
de un sistema es lograr una funcionalidad eficiente y efectiva, y para ello se
utilizan herramientas y metodologías específicas para su diseño, implementación
y gestión.

En el contexto de seguridad e higiene en una empresa, dentro de un 
\textbf{\gls{sgs}}, estos elementos trabajan juntos para lograr una gestión 
efectiva de la seguridad y la salud en el trabajo. Este sistema involucra 
políticas, procedimientos, prácticas y recursos que se utilizan para 
identificar, evaluar y controlar los riesgos asociados a las actividades 
laborales. El objetivo principal de este sistema es prevenir los accidentes 
laborales y las enfermedades profesionales, minimizando los riesgos asociados a
la actividad empresarial y asegurando la salud y el bienestar de los 
trabajadores.

