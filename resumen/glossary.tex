\newacronym{cymat}{CyMAT}{Condiciones y Medio Ambiente de Trabajo}

\newacronym{lrt}{LRT}{Ley de Riesgos del Trabajo}

\newacronym{ppe}{EPP}{Equipos de Protección Personal}

\newacronym{hyst}{HyST}{Higiene y Seguridad en el Trabajo}

\newacronym{sshyst}{SHyST}{Servicio de Higiene y Seguridad en el Trabajo}

\newacronym{oit}{OIT}{Organización Internacional del Trabajo}

\newacronym{srt}{SRT}{Superintendencia de Riesgos de Trabajo}

\newacronym{art}{ART}{Aseguradora de Riesgo de Trabajo}

\longnewglossaryentry{riesgo}{name={riesgo}}{
    Posibilidad de que se produzca un contratiempo o una desgracia, de que alguien o algo sufra perjuicio o daño.
}

\longnewglossaryentry{peligro}{name={peligro}}{
    Situación en la que existe la posibilidad, amenaza u ocasión de que ocurra
    una desgracia o un contratiempo.}

\longnewglossaryentry{probabilidad}{name={probabilidad}}{
    Cálculo matemático de las posibilidades que existen de que una cosa se 
    cumpla o suceda al azar.}

\longnewglossaryentry{factor-de-riesgo}{name={factor de riesgo}}{
    Factor que aumenta la probabilidad de que se produzca un daño, un 
    contratiempo, una desgracia u otra situación negativa, como contraer una 
    enfermedad o sufrir un accidente laboral.}

\longnewglossaryentry{ergonomia}{name={ergonomía}}{
    Estudio de las condiciones de adaptación de un lugar de trabajo, una 
    máquina, un vehículo, etc., a las características físicas y psicológicas 
    del trabajador o el usuario.}

\longnewglossaryentry{infraestructura}{name={infraestructura}}{
    Conjunto de medios técnicos, servicios e instalaciones necesarios para el 
    desarrollo de una actividad o para que un lugar pueda ser utilizado.}

\longnewglossaryentry{hig-trabajo}{name={higiene del trabajo}}{
    Ciencia encargada de anticipar, identificar, evaluar y controlar los 
    riesgos que pueden poner en peligro la salud y el bienestar de los 
    trabajadores.}

\longnewglossaryentry{sistema}{name={sistema}}{
    Conjunto de elementos o componentes interconectados que trabajan juntos 
    para lograr un objetivo común}

\longnewglossaryentry{sgs}{name={sistema de gestión}}{
    Conjunto de elementos interrelacionados que trabajan juntos para lograr una
    gestión efectiva de la seguridad y la salud en el trabajo}

\longnewglossaryentry{accidente-laboral}{name={accidente de trabajo}}{
    Un evento no planificado o no deseado que ocurre en el lugar de trabajo o 
    en el desempeño de una actividad laboral, y que resulta en lesiones, daños 
    a la propiedad o pérdidas.}

\longnewglossaryentry{incidente}{name={incidente}}{
    Un evento no deseado que puede tener el potencial de generar daño, lesiones
    o pérdidas, pero que no ha resultado en un accidente o lesión.}

\longnewglossaryentry{matriz-de-riesgo}{
    name={matriz de análisis y evaluación de riesgos},
    plural={matrices de análisis y evaluación de riesgos}
}{
    Herramienta utilizada en la gestión de la seguridad y salud en el trabajo 
    para identificar, analizar y evaluar los riesgos asociados a una actividad 
    laboral específica. Es una tabla donde en un eje se listan los diferentes 
    tipos de riesgos, y en el otro eje se clasifican los niveles de probabilidad
    y gravedad de dichos riesgos.

    En la matriz se asigna un valor numérico a cada nivel de probabilidad y 
    gravedad, y se utiliza para evaluar y priorizar los riesgos según su 
    importancia relativa. A partir de esto, se pueden tomar decisiones 
    informadas y definir las medidas de prevención y control necesarias para 
    reducir o eliminar los riesgos identificados.}
